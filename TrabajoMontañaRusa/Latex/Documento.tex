\documentclass[11pt,a4paper]{article}

% =========================
% PAQUETES
% =========================
\usepackage[spanish]{babel}
\usepackage[utf8]{inputenc}
\usepackage[T1]{fontenc}
\usepackage{amsmath, amssymb}
\usepackage{graphicx}
\usepackage{float}
\usepackage{booktabs}
\usepackage{geometry}
\usepackage{hyperref}
\usepackage{caption}
\usepackage{subcaption}

\geometry{margin=2.5cm}

% =========================
% DATOS
% =========================
\title{
    \vspace{2cm}
    \textbf{Modelado y simulación de una montaña rusa}\\
    \large Análisis dinámico en 2D y 3D mediante interpolación spline
}
\author{
    Miguel  Enterría Lastra\\
    Grado en Física\\
    Universidad de Oviedo\\
}
\date{\today}

% =========================
% DOCUMENTO
% =========================
\begin{document}

\maketitle
\thispagestyle{empty}
\newpage

\tableofcontents
\newpage

% ============================================================
\section{Introducción}
% ============================================================

En esta práctica se estudia el comportamiento dinámico de un vagón que se desplaza a lo largo de una montaña rusa,
modelada geométricamente mediante curvas paramétricas interpoladas con B-splines.
El estudio se realiza primero en dos dimensiones y posteriormente se generaliza a tres dimensiones,
permitiendo analizar trayectorias más realistas que incluyen giros y torsión.

El objetivo principal es simular el movimiento resolviendo la ecuación diferencial asociada,
comparar distintos métodos numéricos y evaluar magnitudes físicas relevantes como velocidad,
aceleración, fuerza normal y conservación de la energía.

% ============================================================
\section{Modelo físico}
% ============================================================

El vagón se modela como una partícula de masa constante que se desplaza a lo largo de una trayectoria prescrita.
Las fuerzas consideradas son:

\begin{itemize}
    \item Peso.
    \item Reacción normal de la vía.
    \item Fuerza de rozamiento.
    \item Resistencia aerodinámica.
\end{itemize}

La formulación final conduce a un sistema de ecuaciones diferenciales para el parámetro de la curva $u(t)$
y la velocidad $v(t)$, que se resuelve numéricamente mediante \texttt{solve\_ivp}.

% ============================================================
\section{Descripción geométrica del recorrido}
% ============================================================

La trayectoria de la montaña rusa se define a partir de un conjunto de puntos de control,
que posteriormente se interpolan mediante B-splines cúbicos.

\subsection{Trayectoria 2D}

En el caso bidimensional, el movimiento queda restringido a un plano vertical.
La trayectoria incluye elementos básicos como descensos pronunciados, bucles y colinas.
Siguiendo la geometría propeusta por el guión. 

Se muestra a continuación la curva parametrizada de la siguente manera:

\begin{itemize}
    \item Descenso inicial marcado por curva sigmoide.
    \item Loop hecho con un clotoide.
    \item Colina parametrizada como campana de Gauss.
\end{itemize}

Todas las geometrías se han hecho utilizando las funciones presentes en pkgcurvas.py
\begin{figure}[H]
    \centering
    \includegraphics[width=0.8\textwidth]{}
    \caption{Trayectoria interpolada en 2D.}
\end{figure}

\subsection{Trayectoria 3D}

El modelo tridimensional incorpora giros horizontales y tramos no planos.
El recorrido completo se construye concatenando segmentos geométricos suaves.

\begin{figure}[H]
    \centering
    \includegraphics[width=0.8\textwidth]{trayectoria_3D.png}
    \caption{Trayectoria interpolada en 3D.}
\end{figure}

% ============================================================
\section{Simulación numérica}
% ============================================================

La ecuación diferencial se resuelve empleando distintos métodos numéricos,
tanto explícitos como implícitos, con el fin de analizar su estabilidad y precisión.

\subsection{Comparación de métodos}

Para el caso conservativo, se estudia la variación relativa de la energía total.
Un método adecuado debe minimizar la deriva energética.

\begin{figure}[H]
    \centering
    \includegraphics[width=0.8\textwidth]{energia_metodos.png}
    \caption{Variación relativa de la energía para distintos métodos numéricos.}
\end{figure}

\begin{table}[H]
\centering
\caption{Comparación de métodos numéricos}
\begin{tabular}{lcccc}
\toprule
Método & Pasos & $\Delta E_{\min}$ & $\Delta E_{\max}$ & $\sigma(\Delta E)$ \\
\midrule
RK45   &       &                  &                  &                    \\
RK23   &       &                  &                  &                    \\
DOP853 &       &                  &                  &                    \\
Radau  &       &                  &                  &                    \\
BDF    &       &                  &                  &                    \\
LSODA  &       &                  &                  &                    \\
\bottomrule
\end{tabular}
\end{table}

En base a los resultados obtenidos, el método \textbf{Radau} presenta el mejor comportamiento
en términos de conservación de la energía, tanto en 2D como en 3D.

% ============================================================
\section{Resultados}
% ============================================================

A partir de la simulación se obtienen las siguientes magnitudes físicas:

\begin{itemize}
    \item Evolución temporal del parámetro $u(t)$ y de la velocidad $v(t)$.
    \item Fuerza normal ejercida por la vía.
    \item Aceleraciones tangencial y normal.
\end{itemize}

\begin{figure}[H]
    \centering
    \includegraphics[width=\textwidth]{resultados_dinamicos.png}
    \caption{Evolución temporal de las principales magnitudes físicas.}
\end{figure}

Los valores máximos de aceleración y fuerza normal permiten evaluar la viabilidad del diseño
desde el punto de vista de la seguridad.

% ============================================================
\section{Simulación y animación}
% ============================================================

La simulación se complementa con una animación del movimiento del vagón sobre la trayectoria,
lo que permite una interpretación visual del comportamiento dinámico.

En el caso tridimensional, la animación pone de manifiesto la influencia de los giros y la torsión
en la dinámica del sistema.

% ============================================================
\section{Conclusiones}
% ============================================================

Se ha desarrollado un modelo completo para el estudio dinámico de una montaña rusa en 2D y 3D.
El uso de interpolación mediante B-splines permite definir trayectorias suaves y realistas.

La comparación de métodos numéricos muestra que los métodos implícitos,
en particular Radau, ofrecen mayor estabilidad y mejor conservación de la energía.

El análisis de aceleraciones y fuerzas normales indica que el diseño inicial requiere ajustes
para cumplir criterios de seguridad realistas.

% ============================================================
\section*{Anexo: Herramientas empleadas}
% ============================================================

\begin{itemize}
    \item \texttt{NumPy} y \texttt{SciPy} para el cálculo numérico.
    \item \texttt{Matplotlib} para la visualización y animación.
    \item Interpolación mediante \texttt{BSpline}.
\end{itemize}

\end{document}
